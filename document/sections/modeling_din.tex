\section{Modelo Dinâmico}

Para encontrar o modelo dinâmico resultante entre a tensão aplicada nos
motores e as velocidades resultantes sobre o robô, precisamos considerar duas dinâmicas:

\vspace{1em}
A relação entre a tensão aplicada nos motores e o torque resultante sobre as rodas,
em conjunto com as resultantes que esses torques aplicados geram sobre o robô, 
considerando sua massa, momento de inércia e coeficientes de viscosidade.

\subsection{Dinâmica dos Atuadores}

A dinâmica dos atuadores relaciona a tensão aplicada aos motores 
com o torque e a velocidade angular nas rodas, 
resultantes das características do circuito de armadura.

\vspace{1em}
Conforme descrito em \cite{ogata1997modern}, 
as equações diferenciais que representam o circuito de armadura e o equilíbrio de torque em um motor CC são:

\begin{equation}
    u = L \frac{di}{dt} + Ri + K_{\omega} \omega
\end{equation}
\begin{equation}
    t_r = K_{t} i = J \frac{d \omega}{dt} + \beta \omega
\end{equation}

A primeira equação descreve o comportamento elétrico do motor, 
em que a tensão aplicada $u$ se divide entre a indutância $L$ da armadura, 
a resistência $R$, e a força contra-eletromotriz (CEMF) $K_\omega \omega$ 
gerada pela rotação do eixo. Já a segunda equação representa o balanço 
de torques no rotor, onde o torque gerado $K_t i$ é utilizado para vencer 
a inércia do motor $J$ e as perdas por atrito viscoso $\beta$.

\vspace{1em}
Relacionando as duas equações acima, 
podemos eliminar a corrente elétrica $i$ e 
obter uma equação diferencial de segunda ordem que relaciona 
diretamente a tensão aplicada $u$ à velocidade angular $\omega$:

\begin{equation}
    L J \frac{d^2 \omega}{dt^2} + (L \beta + R J) \frac{d \omega}{dt} + (R \beta + K_{\omega} K_{t}) \omega = K_{t} u
\end{equation}

Considerando que $L$ é desprezível, podemos simplificar a equação para:

\begin{equation}
    R J \frac{d \omega}{dt} + (R \beta + K_{\omega} K_{t}) \omega = K_{t} u
\end{equation}

Que, finalmente, pode ser reescrita como:

\begin{equation}
    t_r = \rho K_{t} u - \beta \omega - J \frac{d \omega}{dt}
    \label{eq:atuadores}
\end{equation}

Onde a nomenclatura utilizada é:

\begin{itemize}
    \item $u$, $i$: tensão aplicada e corrente no circuito de armadura;
    \item $L$, $R$, $\rho$: indutância, resistência e o inverso da resistência do motor;
    \item $K_{\omega}$, $\omega$: constante de velocidade angular e velocidade angular;
    \item $K_{t}$, $t_r$: constante de torque e torque resultante;
    \item $J$, $\beta$: momento de inércia e coeficiente de viscosidade do motor.
\end{itemize}

\subsection{Relação de Esforços nos Referenciais}

As forças aplicadas sobre as rodas devido ao torque aplicado nos motores, 
mostradas na Figura \ref{fig:robot_composition}, resultam 
nas velocidades lineares e angulares do robô. 

Em \cite{vieira2005controle}, essas forças são definidas como:

\begin{equation}
f_d = \frac{t_d}{r_d}
\qquad
\qquad
f_e = \frac{t_e}{r_e}
\end{equation}

Assim, resultando nos esforços totais aplicados sobre a estrutura do robô:

\begin{equation}
f = f_d + f_e = \frac{t_d}{r_d} + \frac{t_e}{r_e}
\qquad
\qquad
t = t_d \frac{d}{2r_d} - t_e \frac{d}{2r_e}
\label{eq:esforco}
\end{equation}

Para encontrar esses esforços, precisamos utilizar a equação \ref{eq:atuadores}, 
que define a relação entre os torques aplicados nos motores e as tensões $u_d$ e $u_e$.

Ao escrever a equação dos motores em forma matricial, temos:

\begin{equation}
\begin{bmatrix}
t_d \\
t_e
\end{bmatrix}
=
\begin{bmatrix}
\rho_d K_{t d} & 0 \\
0 & \rho_e K_{t e}
\end{bmatrix}
\begin{bmatrix}
u_d \\
u_e
\end{bmatrix}
-
\begin{bmatrix}
\beta_d & 0 \\
0 & \beta_e
\end{bmatrix}
\begin{bmatrix}
\omega_d \\
\omega_e
\end{bmatrix}
-
\begin{bmatrix}
J_d & 0 \\
0 & J_e
\end{bmatrix}
\begin{bmatrix}
\dot{\omega}_d \\
\dot{\omega}_e
\end{bmatrix}
\end{equation}

Utilizando a equação \ref{eq:velocities}, podemos 
reescrever a relação entre os torques aplicados, 
eliminando as velocidades angulares das rodas $\omega_d$ e $\omega_e$
e substituindo-as pelas velocidades lineares $v$ e $\omega$ do robô:

\begin{equation}
\begin{bmatrix}
\omega_d \\
\omega_e
\end{bmatrix}
= 
\begin{bmatrix}
\frac{1}{r_d} & \frac{d}{2r_d} \\
\frac{1}{r_e} & -\frac{d}{2r_e}
\end{bmatrix}
\begin{bmatrix}
v \\
\omega
\end{bmatrix}
\end{equation}

Resumindo, em forma matricial, 
a relação entre os torques aplicados sobre o robô
e suas velocidades é dada por:

\begin{equation}
\mathbf{t}
=
\mathbf{K}_{mot} \mathbf{u}
-
\mathbf{B}_{mot} \mathbf{\prescript{\omega}{}{T_v}} \mathbf{v}
-
\mathbf{J}_{mot} \mathbf{\prescript{\omega}{}{T_v}} \dot{\mathbf{v}}
\label{eq:force_torque}
\end{equation}

Finalmente, ao aplicarmos a matriz de transformação $\mathbf{\prescript{\text{f}}{}{T_t}}$, 
que projeta os torques das rodas nos eixos do referencial do robô, obtendo
 a equação que relaciona os esforços totais resultantes 
$\mathbf{f}$ com os torques aplicados $\mathbf{t}$, 
encontrando a relação que define as velocidades do robô aos esforços aplicados sobre sua estrutura.

\begin{equation}
\mathbf{f}
=
\mathbf{\prescript{\text{f}}{}{T_t}} \mathbf{t}
\end{equation}

Onde:

\[
\mathbf{t} =
\begin{bmatrix}
t_d \\
t_e
\end{bmatrix}
\qquad
\mathbf{u} =
\begin{bmatrix}
u_d \\
u_e
\end{bmatrix}
\qquad
\mathbf{v} =
\begin{bmatrix}
v \\
\omega
\end{bmatrix}
\qquad
\mathbf{f} =
\begin{bmatrix}
f \\
t
\end{bmatrix}
\qquad
\mathbf{\prescript{\omega}{}{T_v}} =
\begin{bmatrix}
\frac{1}{r_d} & \frac{d}{2r_d} \\
\frac{1}{r_e} & -\frac{d}{2r_e}
\end{bmatrix}
\]

\[
\mathbf{K}_{mot} =
\begin{bmatrix}
\rho_d K_{t d} & 0 \\
0 & \rho_e K_{t e}
\end{bmatrix}
\qquad
\mathbf{B}_{mot} =
\begin{bmatrix}
\beta_d & 0 \\
0 & \beta_e
\end{bmatrix}
\qquad
\mathbf{J}_{mot} =
\begin{bmatrix}
J_d & 0 \\
0 & J_e
\end{bmatrix}
\qquad
\mathbf{\prescript{\text{f}}{}{T_t}} =
\begin{bmatrix}
\frac{1}{r_d} & \frac{1}{r_e} \\
\frac{d}{2r_d} & -\frac{d}{2r_e}
\end{bmatrix}
\]

\subsection{Dinâmica do Robô}

Para encontrar a dinâmica do robô, consideramos as leis de Newton e Euler,
relacionando a força e o torque aplicados sobre o robô com as acelerações linear e angular resultantes.
Assim como em \cite{vieira2005controle}, temos:

\begin{equation}
f = m \dot{v} + \beta_l v
\qquad
\qquad
t = J \dot{\omega} + \beta_{\theta} \omega
\end{equation}

Onde $m$ é a massa do robô, $J$ é o momento de inércia,
$\beta_l$ e $\beta_{\theta}$ são os coeficientes de viscosidade linear e angular, respectivamente.

Substituindo as equações de esforço e torque definidas na equação \ref{eq:esforco}, temos:

\begin{equation}
m \dot{v} + \beta_l v = \frac{t_d}{r_d} + \frac{t_e}{r_e}
\qquad
\qquad
J \dot{\omega} + \beta_{\theta} \omega = t_d \frac{d}{2r_d} - t_e \frac{d}{2r_e}
\end{equation}

Ou em forma matricial:

\begin{equation}
\begin{bmatrix}
m & 0 \\
0 & J
\end{bmatrix}
\begin{bmatrix}
\dot{v} \\
\dot{\omega}
\end{bmatrix}
+
\begin{bmatrix}
\beta_l & 0 \\
0 & \beta_{\theta}
\end{bmatrix}
\begin{bmatrix}
v \\
\omega
\end{bmatrix}
=
\begin{bmatrix}
\frac{1}{r_d} & \frac{1}{r_e} \\
\frac{d}{2r_d} & -\frac{d}{2r_e}
\end{bmatrix}
\begin{bmatrix}
t_d \\
t_e
\end{bmatrix}
\end{equation}

Utilizando a equação \ref{eq:force_torque}, 
podemos reescrever a dinâmica do robô em 
termos da tensão aplicada nos motores:

\begin{equation}
\mathbf{M}_{robo} \dot{\mathbf{v}} + \mathbf{B}_{robo} \mathbf{v} = 
\mathbf{\prescript{\text{f}}{}{T_t}} (\mathbf{K}_{mot} \mathbf{u} - \mathbf{B}_{mot} \mathbf{\prescript{\omega}{}{T_v}} \mathbf{v} - \mathbf{J}_{mot} \mathbf{\prescript{\omega}{}{T_v}} \dot{\mathbf{v}})
\end{equation}

Deixando $\mathbf{v}$ em função de $\mathbf{u}$, temos:

\begin{equation}
(\mathbf{M}_{robo} + \mathbf{\prescript{\text{f}}{}{T_t}} \mathbf{J}_{mot} \mathbf{\prescript{\omega}{}{T_v}}) \dot{\mathbf{v}}
+
(\mathbf{B}_{robo} + \mathbf{\prescript{\text{f}}{}{T_t}} \mathbf{B}_{mot} \mathbf{\prescript{\omega}{}{T_v}}) \mathbf{v}
=
\mathbf{\prescript{\text{f}}{}{T_t}} \mathbf{K}_{mot} \mathbf{u}
\end{equation}

Onde $\mathbf{M}$ é a matriz de massa e inércia do robô e 
$\mathbf{B}$ é a matriz de viscosidade, representadas por:

\[
\mathbf{M}_{robo} =
\begin{bmatrix}
m & 0 \\
0 & J
\end{bmatrix}
\qquad
\mathbf{B}_{robo} =
\begin{bmatrix}
\beta_l & 0 \\
0 & \beta_{\theta}
\end{bmatrix}
\]

Por fim, a dinâmica do robô pode ser expressa como:

\begin{equation}
\mathbf{M} \dot{\mathbf{v}} + \mathbf{B} \mathbf{v} = \mathbf{K} \mathbf{u}
\label{eq:robot_dynamics}
\end{equation}

Onde:
\[
\mathbf{M} = \mathbf{M}_{robo} + \mathbf{\prescript{\text{f}}{}{T_t}} \mathbf{J}_{mot} \mathbf{\prescript{\omega}{}{T_v}}
\qquad
\mathbf{B} = \mathbf{B}_{robo} + \mathbf{\prescript{\text{f}}{}{T_t}} \mathbf{B}_{mot} \mathbf{\prescript{\omega}{}{T_v}}
\qquad
\mathbf{K} = \mathbf{\prescript{\text{f}}{}{T_t}} \mathbf{K}_{mot}
\]

O sistema apresenta um comportamento dinâmico linear de 
primeira ordem. Assumindo que $\mathbf{M}$ 
e $\mathbf{B}$ são matrizes invertíveis e constantes, 
a equação pode ser reescrita em espaços de estados como:

\begin{equation}
\begin{cases}
\dot{\mathbf{v}} = -\mathbf{M}^{-1} \mathbf{B} \mathbf{v} + \mathbf{M}^{-1} \mathbf{K} \mathbf{u} \\
\dot{\mathbf{q}} = \mathbf{\prescript{v}{}{T_{(x, y)}}} \mathbf{v}
\end{cases}
\end{equation}

Onde $\mathbf{q}$ é o vetor de estados do robô.

\[
\mathbf{\prescript{v}{}{T_{(x, y)}}} =
\begin{bmatrix}
\cos(\theta) & 0 \\
\sin(\theta) & 0 \\
0 & 1
\end{bmatrix}
\qquad
\mathbf{q} =
\begin{bmatrix}
x \\
y \\
\theta
\end{bmatrix}
\qquad
\dot{\mathbf{q}} =
\begin{bmatrix}
\dot{x} \\
\dot{y} \\
\dot{\theta}
\end{bmatrix}
\]
