\section{Controle Preditivo Generalizado (GPC)}

O Controle Preditivo Generalizado (GPC) é uma 
técnica avançada de controle preditivo (MPC) 
que adota a função de transferência do sistema 
como modelo. No qual, nesse trabalho a função de transferência é proveniente 
da equação descrita em \ref{eq:robot_dynamics}.

\vspace{1em}
Nota-se que o sistema descrito possui um comportamento dinâmico linear de
primeira ordem, onde é possivel encontrar as matrizes que contêm 
as constantes de tempo e ganhos estáticos do sistema:
\begin{equation}
\boldsymbol{\tau} = \mathbf{B}^{-1} \mathbf{M} \\
\qquad
\qquad
\mathbf{K}_u = \mathbf{B}^{-1} \mathbf{K}
\end{equation}

Permitindo que encontremos a função de transferência das variáveis 
de controle para cada uma das variáveis de saída do sistema. 
Que ao utilizarmos $u_d$, $u_e$, $v$ e $\omega$,
no domínio discreto, usando a transformada $z$, elas são dadas por:
\[
\mathbf{H}_{d v}(z^{-1}) = \frac{b_{d v} z^{-1}}{1 - a_{d} z^{-1}} \\
\qquad
\mathbf{H}_{d \omega}(z^{-1}) = \frac{b_{d \omega} z^{-1}}{1 - a_{d} z^{-1}} \\
\]
\[
\mathbf{H}_{e v}(z^{-1}) = \frac{b_{e v} z^{-1}}{1 - a_{e} z^{-1}} \\
\qquad
\mathbf{H}_{e \omega}(z^{-1}) = \frac{b_{e \omega} z^{-1}}{1 - a_{e} z^{-1}}
\]

Onde os parâmetros $b$ e $a$ são obtidos a partir da discretização 
da constante de tempo e ganho estático extraídos das matrizes
$\mathbf{K}$ e $\boldsymbol{\tau}$.

\vspace{1em}
Integrando os valores das velocidades para obter as posições 
(sendo $T$ o período de amostragem) e utilizando os incrementos 
de controle $\Delta u$, conforme a implementação típica do GPC, 
obtêm-se as funções de transferência discretas para cada variável 
de controle e saída do sistema $\frac{Y(z)}{\Delta U(z)}$.

\[
\mathbf{H}_{d s}(z^{-1}) = \frac{b_{d v} z^{-1}}{1 - a_{d} z^{-1}} \; \frac{T}{(1 - z^{-1})^2}\\
\qquad
\mathbf{H}_{d \theta}(z^{-1}) = \frac{b_{d \theta} z^{-1}}{1 - a_{d} z^{-1}} \; \frac{T}{(1 - z^{-1})^2}\\
\]
\[
\mathbf{H}_{e s}(z^{-1}) = \frac{b_{e v} z^{-1}}{1 - a_{e} z^{-1}} \; \frac{T}{(1 - z^{-1})^2}\\
\qquad
\mathbf{H}_{e \theta}(z^{-1}) = \frac{b_{e \theta} z^{-1}}{1 - a_{e} z^{-1}} \; \frac{T}{(1 - z^{-1})^2}
\]

Por fim, a partir das funções de transferência do sistema,
podemos definir o modelo que será utilizado no GPC.
Nesse modelo, a matriz $G$ representa a resposta ao degrau para cada variável de controle,
enquanto $F$ é a matriz de polinômios responsável pelo cálculo da
resposta livre a partir de $y[k]$.

Assim podemos definir a predição futura como: 

\begin{equation}
\hat{\mathbf{y}} = \mathbf{G} \Delta \mathbf{u} + \mathbf{F}(z^{-1}) \mathbf{y}[k]
\end{equation}

Resultando na função de custo do GPC, em que $\mathbf{w}$ representa o valor de 
referência, e $\boldsymbol{\delta}$ e $\boldsymbol{\lambda}$ são os pesos associados, 
respectivamente, à ação de controle e ao erro de predição."

\begin{equation}
J_u = \boldsymbol{\delta} (\hat{\mathbf{y}} - \mathbf{w})^2 + \boldsymbol{\lambda} \mathbf{\Delta u}^2
\end{equation}

\[
\mathbf{G} = 
\begin{bmatrix}
\mathbf{G}_{d s} & \mathbf{G}_{e s}\\
\mathbf{G}_{d \theta}  & \mathbf{G}_{e \theta}
\end{bmatrix}
\quad
\mathbf{\Delta u} =
\begin{bmatrix}
\mathbf{\Delta u_d} \\
\mathbf{\Delta u_e}
\end{bmatrix}
\quad
\mathbf{F}(z^{-1}) =
\begin{bmatrix}
\mathbf{F}_{s}(z^{-1}) & 0\\
0 & \mathbf{F}_{\theta}(z^{-1})
\end{bmatrix}
\quad
\mathbf{y}[k] =
\begin{bmatrix}
s[k] \\
\theta[k]
\end{bmatrix}
\]

Utilizando o modelo linearizado descrito na equação \ref{eq:linearized_model},
podemos relacionar $s$ e $\theta$ diretamente com $(x, y)$.